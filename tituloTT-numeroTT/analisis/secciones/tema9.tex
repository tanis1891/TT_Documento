\newpage\section{An{á}lisis sensor de Pulso Card{í}aco}

Para comprender mejor el concepto de frecuencia cardíaca, empezaremos por precisar su definición posteriormente daremos paso a explicar las causas que afectan y los diferentes métodos de medición. Sin embargo, también es importante mencionar que a medida que se envejece, el deterioro fisiológico normal y la presencia de enfermedades, disminuye progresivamente la capacidad funcional. \\

La frecuencia cardíaca (FC), se define como el número de contracciones ventriculares efectuadas por el corazón, medida generalmente en latidos por minuto $(lat \bullet min^{-1})$ o pulsaciones por minuto (ppm), de tal forma que el pulso puede ser palpable en cualquier arteria \cite{cincuentayuno}. \\

El pulso es uno de los parámetros que representa la expresión periférica de la actividad del corazón. En el adulto, la frecuencia cardíaca (pulso) normal oscila entre 60 y 100 por minuto, menos de 60 se considera bradicardia la cual es extrema si el valor es inferior a 30 por minuto, más de 100 pulsaciones se considera taquicardia y es severa si sobrepasa los 170 por minuto, la severidad está determinada porque las cifras que sobrepasan estos rangos, casi siempre se asocian a síntomas de bajo gasto cardíaco (hipotensión, mareos, síncope, etc.) \cite{cincuentaydos}. \\

En las tablas 2.35, 2.36 y 2.37, se muestran los valores normales de la frecuencia cardíaca en reposo clasificados por edades y sexo.

\begin{table}
	\begin{tabular}{|p{4cm}|p{4cm}|p{4cm}|}
		\hline 
		\centering Clasificación & \centering Mujeres & \centering Hombres \\ 
		\tabularnewline
		\hline 
		Excelente & \centering $\leq$53 & \centering $\leq$56 \\ 
		\tabularnewline
		\hline 
		Bueno & \centering $60-64$ & \centering $64-57$ \\ 
		\tabularnewline
		\hline 
		Promedio & \centering $65-61$ & \centering $71-65$ \\  
		\tabularnewline
		\hline 
		Pobre & \centering $75-66$ & \centering $79-72$ \\ 
		\tabularnewline
		\hline 
		Muy Pobre & \centering $\geq$76 & \centering$\geq$80 \\ 
		\tabularnewline
		\hline 
	\end{tabular} 
	\textbf{\caption{\small{\textbf{Escala de clasificación para la frecuencia cardíaca en reposo de mujeres y hombres (latidos por minuto) \cite{cincuentaytres}}}}}
\end{table}

\newpage
\subsection{Etapas de una caída}


