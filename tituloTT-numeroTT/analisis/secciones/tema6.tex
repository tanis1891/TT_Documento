\newpage\section{An{á}lisis de Riesgos}

En esta sección hablaremos de los riesgos que pueden afectar el desarrollo del prototipo, la probabilidad que de estos ocurran, y el impacto que tienen en el prototipo. El análisis y la gestión de riesgos son una serie de pasos que nos permiten comprender y gestionar la incertidumbre que se tiene en el prototipo \cite{cuarentaycuatro}. \\

Comenzaremos definiendo lo que es un riesgo para poder identificarlos, un riesgo es un problema potencial que afecta al desarrollo del proyecto, este puede pasar o no. \\

Los pasos que realizaremos para analizar y gestionar los riesgos son los siguientes: \\

\begin{itemize}
	\item Identificarlos.
	\item Evaluar la probabilidad de ocurrencia.
	\item Estimar el impacto que tendrían.
	\item Establecer un plan de contingencia para los riesgos de alto impacto.	
\end{itemize}

Los riesgos que se identificaron son los siguientes: \\

\begin{itemize}
	\item R1: Mala organización con los integrantes del equipo.
	\item R2: El usuario no está capacitado para utilizar el producto.
	\item R3: Un integrante del equipo sufra alguna enfermedad grave.
	\item R4: Los sensores no se encuentren en óptimas condiciones a la hora de las pruebas.
	\item R5: La comunicación entre los sensores y el microcontrolador no sea correcta.
	\item R6: El módulo de Wifi no transmita bien los datos.
	\item R7: La comunicación entre la aplicación y hardware no son correctos.
	\item R8: El prototipo no se termine en el tiempo estimado.
	\item R9: El dispositivo móvil no cuenta con las características especificadas para el proyecto.
	\item R10: El usuario directo arruine la pulsera de sensores.
	\item R11: Se presente un inconveniente en el desarrollo por falta de conocimiento.
	\item R12: La alimentación suministrada por las pilas hacia los sensores no sea suficiente para realizar las pruebas mínimas.
	\item R13: El microcontrolador se arruine.
	\item R14: El prototipo no cumpla las expectativas del cliente.
	\item R15: Los sensores no se encuentre calibrados.
\end{itemize}

Una vez que se identificaron los riesgos entre el equipo de trabajo, lo siguiente es estimar la probabilidad de que ocurran, esta probabilidad se toma de los integrantes del equipo pues cada uno tiene en mente las actividades que se deben desarrollar, y los problemas que se pueden presentar. Además, que se identifican los riesgos de acuerdo, al tipo de riesgo y al nivel de impacto que tienen \cite{cuarentaycuatro}. \\

\textbf{Tipo de riesgo} \\

\textbf{Riesgo de rendimiento (Rr):} El grado de incertidumbre con el que el producto encontrará sus requisitos y se adecue para su empleo pretendido. \\

\textbf{Riesgo de coste (Rc):} El grado de incertidumbre que mantendrá el presupuesto del proyecto. \\

\textbf{Riesgo de soporte (Rs):} El grado de incertidumbre de la facilidad del software para corregirse, adaptarse y ser mejorado. \\

\textbf{Riesgo de la planificación temporal (Rpt):} El grado de incertidumbre con que se podrá mantener la planificación temporal y de que el producto se entregue a tiempo. \\

El impacto se considera de acuerdo a los siguientes valores: \\

\begin{enumerate}
	\item Catastrófico.
	\item Critico.
	\item Marginal.
	\item Despreciable.
\end{enumerate}

\begin{longtable}{|p{1cm}|p{1.5cm}|p{2cm}|p{1.2cm}|p{5.5cm}|}
%	\centering
		\hline
		\centering Riesgo & \centering Categoría & \centering Probabilidad & \centering Impacto	& \centering Plan de contingencia \\
		\tabularnewline
		\hline 
		\centering R1 & \centering Rpt & \centering 20\% & \centering 2	& Evaluar las actividades en equipo y asignar actividades faltantes, de acuerdo al cronograma \\
		\hline 
		\centering R2 & \centering Rr & \centering 30\% & \centering 3 & Se realizarán manuales o guías que expliquen el funcionamiento y uso del prototipo \\
		\hline 
		\centering R3 & \centering Rpt & \centering 10\% & \centering 2 & Se dividirá las actividades faltantes con los únicos integrantes del equipo \\
		\hline
		\centering R4 & \centering Rs & \centering 50\% & \centering 3 & Se realizarán pruebas mínimas antes de realizar las pruebas finales, se tendrán repuesto en el caso de que sea necesario cambiarlos \\
		\hline
		\centering R5 & \centering Rs & \centering 30\% & \centering 1 & Se realizarán pruebas entre cada sensor y el microcontrolador, para determinar la posible falla \\
		\hline
		\centering R6 & \centering Rs & \centering 40\% & \centering 2 & Se realizarán pruebas con el módulo de manera individual, y se observarán las posibles fallas que este pueda presentar, para realizar buenas prácticas \\ 
		\hline
		\centering R7 & \centering Rs & \centering 40\% & \centering 1 & Se analizará el hardware de manera individual, una vez que se determine si los resultados son los esperados, evaluaremos la aplicación de acuerdo módulos por los que pasan los datos \\ 
		\hline
		\centering R8 & \centering Rpt & \centering 40\% & \centering 1 & Se organizarán juntas, en las que se evaluara el avance del prototipo y las actividades faltantes, para determinar un plan de acción, para regularizar los tiempos con el cronograma \\ 
		\hline
		\centering R9 & \centering Rs & \centering 20\% & \centering 2 & Se indicará las tecnologías requeridas para el funcionamiento del prototipo \\ 
		\hline
		\centering R10 & \centering Rr & \centering 10\% & \centering 2 & Se pondrán en el manual de usuario, acciones que pongan en riesgo el funcionamiento del prototipo, con el fin de que se tomen en cuenta \\ 
		\hline
		\centering R11 & \centering Rpt & \centering 30\% & \centering 1 & Se realizarán investigaciones previas para dominar el tema.
		Se realizarán juntas con el fin de discutir los temas desconocidos, para después tomar cursos o talleres que nos permitan cubrir ese conocimiento \\ 
		\hline
		\centering R12 & \centering Rr & \centering 30\% & \centering 4 & Se realizarán pruebas una vez que se tenga implementado todo el prototipo, con el fin de analizar la corriente que esté consume, y determinar el arreglo de pilas que se debe utilizar \\ 
		\hline
		\centering R13 & \centering Rr & \centering 10\% & \centering 2 & Comprar varios repuestos para poder cambiarlos, además de analizar el manual para determinar buenas prácticas \\ 
		\hline
		\centering R14 & \centering Rr & \centering 20\% & \centering 2 & Se pensará en las dificultades del cliente indirecto, con el fin de reducir las molestias que pudiese presentar el prototipo \\ 
		\hline
		\centering R15 & \centering Rr & \centering 10\% & \centering 4 & Se implementará un estudio del usuario final con el fin de estudiar las variables, que pudiese presentar su condición actual, se considerará el factor de emisión su frecuencia cardíaca acorde a su edad \\ 
		\hline	
		\caption{Análisis de riesgos \cite{cuarentaycuatro}.}
\end{longtable}

Una vez que se tienen los riesgos se evalúa el impacto y la probabilidad de cada riesgo, mediante la Tabla 2.XVI \\

\begin{table}[htb]
	\setlength{\arrayrulewidth}{0.3mm}

	\begin{tabular}{|p{2cm}|p{1.5cm}|p{2cm}|p{1.6cm}|p{1.6cm}|p{2cm}|}
%		\centering
		\hline
		\cellcolor{blue} \color{white} & \cellcolor[gray]{0.9}\centering Impacto & \cellcolor[gray]{0.9}\centering Despreciable & \cellcolor[gray]{0.9}\centering Marginal & \cellcolor[gray]{0.9}\centering Critico & \cellcolor[gray]{0.9}Catastrófico\\
		\cline{2-6}
		\cellcolor{blue} \color{white} \multirow{2}{2cm}[5mm]{Probabilidad} & & \centering 4 & \centering 3 & \centering 2 & \centering 1 \\
		\tabularnewline
		\hline
		\cellcolor{blue} \color{white} Raro & \centering 1 & \cellcolor{green} \centering R15 & \cellcolor{green} \centering R2,R9,R13, & \cellcolor{yellow} \centering R1, R3, R10, R13,R14, & \cellcolor{red} \\
		\hline
		\cellcolor{blue} \color{white} Moderado & \centering 2 & \cellcolor{yellow} \centering R12 & \cellcolor{yellow} & \cellcolor{red} & \cellcolor{red} \centering R11, R5 \\
		\tabularnewline
		\hline
		\cellcolor{blue} \color{white} Muy probable & \centering 3 & \cellcolor{yellow} & \cellcolor{yellow} \centering R4 & \cellcolor{red} \centering R6 & \cellcolor{red} \centering R7,R8, \\
		\tabularnewline
		\hline
		\cellcolor{blue} \color{white} Seguro & \centering 4 & \cellcolor{yellow} & \cellcolor{yellow} & \cellcolor{red} & \cellcolor{red} \\
		\hline
	\end{tabular}
\textbf{\caption{\small{\textbf{Semáforo de riesgos \cite{cuarentaycuatro}.}}}}
\end{table}

En parte de las filas los riesgos se ordenan conforme a la probabilidad de ocurrencia y en las columnas se ordenan de acuerdo a su impacto, como resultado podemos ver los riesgos más peligrosos en la parte de rojo, y son los que tendremos que impedir o responder, para disminuir su impacto en el prototipo, mediante los planes de contingencia. \\

\textbf{Planes de contingencia}

En este apartado describiremos los planes de contingencia mediante una hoja de información, para ser tomados en cuenta en caso de que lleguen a ocurrir, y de esta manera eliminar el riesgo o en el peor de los casos mitigarlo. \\

%---------------------------- Tabla de riesgo 1 ----------------------% 
\begin{table}[h!]
	\centering
	\begin{tabular}{|p{2.75cm}|p{2.75cm}|p{2.75cm}|p{2.75cm}|}
		\hline
		\multicolumn{4}{|c|}{Hoja de información de riesgo}\\
		\hline
		\multicolumn{1}{|c|}{ID: \textbf{R1}} &
		\multicolumn{1}{|c|}{Fecha: 25/Octubre/2016} &
		\multicolumn{1}{|c|}{Probabilidad: 20\%} &
		\multicolumn{1}{|c|}{Impacto: Critico} \\
		\hline
		\multicolumn{4}{|c|}{Descripción: Mala organización con los integrantes del equipo.}\\
		\hline
		\multicolumn{4}{|p{11cm}|}{\textbf{Refinamiento/contexto:} \par 
			\parbox[p][0.2\textwidth][c]{12cm}{
				\begin{itemize}
					\item Subcondición 1: No tener definido los temas que se deben desarrollar. 
					\item Subcondición 2: Repetición de información entre los integrantes del equipo, llegando a perder el tiempo de desarrollo.
				\end{itemize}		
			}		
		} \\
		\hline
		\multicolumn{4}{|p{11cm}|}{\textbf{Reducción/supervisión:} \par
			\parbox[p][0.27\textwidth][c]{12cm}{
				\begin{enumerate}
					\item Se agendarán juntas semanales que permitan evaluar el avance del prototipo.
					\item En las juntas se organizará el trabajo con forme al cronograma.
					\item Si los procedimientos son de un alto grado de dificultad se atenderá entre el equipo hasta que se resuelva.
				\end{enumerate}		
			}	
		}\\
		\hline
		\multicolumn{4}{|p{12cm}|}{\textbf{Gestión/Plan de contingencia/acción:} Realizar revisiones semanalmente del avance y actividades pendientes.}\\
		\hline
	\end{tabular}
\textbf{\caption{\small{\textbf{Tabla de datos de riesgo 1 \cite{cuarentaycuatro}.}}}}
\end{table}
	
%---------------------------- Tabla de riesgo 2 ----------------------% 	
\begin{table}[h!]
	\centering
	\begin{tabular}{|p{2.75cm}|p{2.75cm}|p{2.75cm}|p{2.75cm}|}
		\hline
		\multicolumn{4}{|c|}{Hoja de información de riesgo}\\
		\hline
		\multicolumn{1}{|c|}{ID: \textbf{R2}} &
		\multicolumn{1}{|c|}{Fecha: 25/Octubre/2016} &
		\multicolumn{1}{|c|}{Probabilidad: 30\%} &
		\multicolumn{1}{|c|}{Impacto: Critico} \\
		\hline
		\multicolumn{4}{|c|}{Descripción: El usuario no está capacitado para utilizar el producto.}\\
		\hline
		\multicolumn{4}{|p{11cm}|}{\textbf{Refinamiento/contexto:} \par 
			\parbox[p][0.17\textwidth][c]{12cm}{
				\begin{itemize}
					\item Subcondición 1: No se tiene información sobre el prototipo. 
					\item Subcondición 2: Se desconoce cómo utilizar el prototipo.
				\end{itemize}		
			}		
		} \\
		\hline
		\multicolumn{4}{|p{11cm}|}{\textbf{Reducción/supervisión:} \par
			\parbox[p][0.2\textwidth][c]{12cm}{
				\begin{enumerate}
					\item Se realizará detalladamente un manual de usuario con el fin de explicar el funcionamiento del prototipo.
					\item En el manual también indicara los límites del prototipo, y los alcances que este puede tener.
				\end{enumerate}		
			}	
		}\\
		\hline
		\multicolumn{4}{|p{12cm}|}{\textbf{Gestión/Plan de contingencia/acción:} Se realizarán manuales o guías de usuario enfocadas al usuario final, explicando el funcionamiento y alcances del prototipo.}\\
		\hline
	\end{tabular}
	\textbf{\caption{\small{\textbf{Tabla de datos de riesgo 2 \cite{cuarentaycuatro}.}}}}
\end{table}

%---------------------------- Tabla de riesgo 3 ----------------------% 
\begin{table}[h!]
	\centering
	\begin{tabular}{|p{2.75cm}|p{2.75cm}|p{2.75cm}|p{2.75cm}|}
		\hline
		\multicolumn{4}{|c|}{Hoja de información de riesgo}\\
		\hline
		\multicolumn{1}{|c|}{ID: \textbf{R3}} &
		\multicolumn{1}{|c|}{Fecha: 25/Octubre/2016} &
		\multicolumn{1}{|c|}{Probabilidad: 10\%} &
		\multicolumn{1}{|c|}{Impacto: Critico} \\
		\hline
		\multicolumn{4}{|c|}{Descripción: Un integrante del equipo sufra alguna enfermedad grave.}\\
		\hline
		\multicolumn{4}{|p{11cm}|}{\textbf{Refinamiento/contexto:} \par 
			\parbox[p][0.2\textwidth][c]{12cm}{
				\begin{itemize}
					\item Subcondición 1: Un integrante del equipo llegue a enfermarse por motivos del entorno.
					\item Subcondición 2: Un integrante se enferme por motivos de estrés o cualquier motivo.
				\end{itemize}		
			}		
		} \\
		\hline
		\multicolumn{4}{|p{11cm}|}{\textbf{Reducción/supervisión:} \par
			\parbox[p][0.2\textwidth][c]{12cm}{
				\begin{enumerate}
					\item Se analizará el avance del trabajo y los pendientes.
					\item Se reasignarán las tareas que estén pendientes entre los integrantes restantes, con el fin de terminar en tiempo y forma.
				\end{enumerate}		
			}	
		}\\
		\hline
		\multicolumn{4}{|p{12cm}|}{\textbf{Gestión/Plan de contingencia/acción:} Se dividirá las actividades restantes entre los integrantes actuales del equipo.}\\
		\hline
	\end{tabular}
	\textbf{\caption{\small{\textbf{Tabla de datos de riesgo 3 \cite{cuarentaycuatro}.}}}}
\end{table}

%---------------------------- Tabla de riesgo 4 ----------------------% 
\begin{table}[htb]
	\centering
	\begin{tabular}{|p{2.75cm}|p{2.75cm}|p{2.75cm}|p{2.75cm}|}
		\hline
		\multicolumn{4}{|c|}{Hoja de información de riesgo}\\
		\hline
		\multicolumn{1}{|c|}{ID: \textbf{R4}} &
		\multicolumn{1}{|c|}{Fecha: 25/Octubre/2016} &
		\multicolumn{1}{|c|}{Probabilidad: 50\%} &
		\multicolumn{1}{|c|}{Impacto: Marginal} \\
		\hline
		\multicolumn{4}{|p{12cm}|}{Descripción: Los sensores no se encuentren en óptimas condiciones a la hora de las pruebas.}\\
		\hline
		\multicolumn{4}{|p{11cm}|}{\textbf{Refinamiento/contexto:} \par 
			\parbox[p][0.2\textwidth][c]{12.3cm}{
				\begin{itemize}
					\item Subcondición 1: Se presenta un corto circuito y se arruinan los sensores.
					\item Subcondición 2: Se utilizan de manera incorrecta los sensores, llegando a confundir los pines.
				\end{itemize}		
			}		
		} \\
		\hline
		\multicolumn{4}{|p{11cm}|}{\textbf{Reducción/supervisión:} \par
			\parbox[p][0.3\textwidth][c]{12.3cm}{
				\begin{enumerate}
					\item Se comprarán repuestos de los sensores con el fin de tener un extra por si lleva a arruinarse alguno.
					\item Se analizará el manual entre los integrantes del equipo, con el fin de tener más percepción de los errores y unificar el conocimiento entre el equipo.
					\item Cuando se realicen pruebas se comprobará si la conexión es la correcta.
				\end{enumerate}		
			}	
		}\\
		\hline
		\multicolumn{4}{|p{12.3cm}|}{\textbf{Gestión/Plan de contingencia/acción:} Se realizarán pruebas mínimas antes de realizar las pruebas finales, se tendrán repuesto en el caso de que sea necesario cambiarlos.}\\
		\hline
	\end{tabular}
	\textbf{\caption{\small{\textbf{Tabla de datos de riesgo 4 \cite{cuarentaycuatro}.}}}}
\end{table}

%---------------------------- Tabla de riesgo 5 ----------------------% 
\begin{table}[htb]
	\centering
	\begin{tabular}{|p{2.75cm}|p{2.75cm}|p{2.75cm}|p{2.75cm}|}
		\hline
		\multicolumn{4}{|c|}{Hoja de información de riesgo}\\
		\hline
		\multicolumn{1}{|c|}{ID: \textbf{R5}} &
		\multicolumn{1}{|c|}{Fecha: 25/Octubre/2016} &
		\multicolumn{1}{|c|}{Probabilidad: 30\%} &
		\multicolumn{1}{|c|}{Impacto: Catastrófico} \\
		\hline
		\multicolumn{4}{|p{12cm}|}{Descripción: La comunicación entre los sensores y el microcontrolador no sea correcta.}\\
		\hline
		\multicolumn{4}{|p{11cm}|}{\textbf{Refinamiento/contexto:} \par 
			\parbox[p][0.16\textwidth][c]{12.3cm}{
				\begin{itemize}
					\item Subcondición 1: El protocolo de comunicación no está bien elaborado.
					\item Subcondición 2: Los datos no son entendibles entre el sensor y el microcontrolador.
				\end{itemize}		
			}		
		} \\
		\hline
		\multicolumn{4}{|p{11cm}|}{\textbf{Reducción/supervisión:} \par
			\parbox[p][0.23\textwidth][c]{12.3cm}{
				\begin{enumerate}
					\item Se asignará tiempo para analizar el protocolo de comunicación.
					\item Se estudiará el funcionamiento de la comunicación de los sensores.
					\item Se realizarán prácticas con el fin de ir incrementando la complejidad de los datos, hasta llegar al objetivo, estudiando cada caso.
				\end{enumerate}		
			}	
		}\\
		\hline
		\multicolumn{4}{|p{12.3cm}|}{\textbf{Gestión/Plan de contingencia/acción:} Se realizarán pruebas entre cada sensor y el microcontrolador, para determinar la posible falla.}\\
		\hline
	\end{tabular}
	\textbf{\caption{\small{\textbf{Tabla de datos de riesgo 5 \cite{cuarentaycuatro}.}}}}
\end{table}

%---------------------------- Tabla de riesgo 6 ----------------------% 
\begin{table}[htb]
	\centering
	\begin{tabular}{|p{2.75cm}|p{2.75cm}|p{2.75cm}|p{2.75cm}|}
		\hline
		\multicolumn{4}{|c|}{Hoja de información de riesgo}\\
		\hline
		\multicolumn{1}{|c|}{ID: \textbf{R6}} &
		\multicolumn{1}{|c|}{Fecha: 25/Octubre/2016} &
		\multicolumn{1}{|c|}{Probabilidad: 40\%} &
		\multicolumn{1}{|c|}{Impacto: Critico} \\
		\hline
		\multicolumn{4}{|c|}{Descripción: El módulo de Wifi no transmite bien los datos.}\\
		\hline
		\multicolumn{4}{|p{11cm}|}{\textbf{Refinamiento/contexto:} \par 
			\parbox[p][0.13\textwidth][c]{12cm}{
				\begin{itemize}
					\item Subcondición 1: Los datos no son transmitidos por el módulo wifi. 
					\item Subcondición 2: La configuración del módulo wifi no es correcto.
				\end{itemize}		
			}		
		} \\
		\hline
		\multicolumn{4}{|p{11cm}|}{\textbf{Reducción/supervisión:} \par
			\parbox[p][0.3\textwidth][c]{12cm}{
				\begin{enumerate}
					\item Se asignará tiempo para analizar el protocolo de comunicación Wifi.
					\item Se realizarán pruebas simples con el módulo Wifi, hasta llegar al objetivo.
					\item Se realizarán prácticas con el fin de ir incrementando la complejidad, hasta llegar al objetivo, estudiando cada caso.
				\end{enumerate}		
			}	
		}\\
		\hline
		\multicolumn{4}{|p{12cm}|}{\textbf{Gestión/Plan de contingencia/acción:} Se realizarán pruebas con el módulo de manera individual, y se observarán las posibles fallas que este pueda presentar, para realizar buenas prácticas.}\\
		\hline
	\end{tabular}
	\textbf{\caption{\small{\textbf{Tabla de datos de riesgo 6 \cite{cuarentaycuatro}.}}}}
\end{table}

%---------------------------- Tabla de riesgo 7 ----------------------% 
\begin{table}[htb]
	\centering
	\begin{tabular}{|p{2.75cm}|p{2.75cm}|p{2.75cm}|p{2.75cm}|}
		\hline
		\multicolumn{4}{|c|}{Hoja de información de riesgo}\\
		\hline
		\multicolumn{1}{|c|}{ID: \textbf{R7}} &
		\multicolumn{1}{|c|}{Fecha: 25/Octubre/2016} &
		\multicolumn{1}{|c|}{Probabilidad: 40\%} &
		\multicolumn{1}{|c|}{Impacto: Catastrófico} \\
		\hline
		\multicolumn{4}{|p{12.3cm}|}{Descripción: La comunicación entre la aplicación y hardware no son correctos.}\\
		\hline
		\multicolumn{4}{|p{11cm}|}{\textbf{Refinamiento/contexto:} \par 
			\parbox[p][0.16\textwidth][c]{12.3cm}{
				\begin{itemize}
					\item Subcondición 1: Los datos no se pueden mostrar en la aplicación móvil. 
					\item Subcondición 2: Se tienen caracteres extraños en la aplicación. 
				\end{itemize}		
			}		
		} \\
		\hline
		\multicolumn{4}{|p{11cm}|}{\textbf{Reducción/supervisión:} \par
			\parbox[p][0.2\textwidth][c]{12.3cm}{
				\begin{enumerate}
					\item Se investigará los protocolos de comunicación que sean compatibles entre el microcontrolador y el IDE de la aplicación, mediante librerías.
					\item Una vez que se tengan los protocolos compatibles, se analizara la estructura de datos transmitida, para ser procesada.
				\end{enumerate}		
			}	
		}\\
		\hline
		\multicolumn{4}{|p{12.3cm}|}{\textbf{Gestión/Plan de contingencia/acción:} Se analizará el hardware de manera individual, una vez que se determine si los resultados son los esperados, evaluaremos la aplicación de acuerdo módulos por los que pasan los datos.}\\
		\hline
	\end{tabular}
	\textbf{\caption{\small{\textbf{Tabla de datos de riesgo 7 \cite{cuarentaycuatro}.}}}}
\end{table}

%---------------------------- Tabla de riesgo 8 ----------------------% 
\begin{table}[htb]
	\centering
	\begin{tabular}{|p{2.75cm}|p{2.75cm}|p{2.75cm}|p{2.75cm}|}
		\hline
		\multicolumn{4}{|c|}{Hoja de información de riesgo}\\
		\hline
		\multicolumn{1}{|c|}{ID: \textbf{R8}} &
		\multicolumn{1}{|c|}{Fecha: 25/Octubre/2016} &
		\multicolumn{1}{|c|}{Probabilidad: 40\%} &
		\multicolumn{1}{|c|}{Impacto: Catastrófico} \\
		\hline
		\multicolumn{4}{|c|}{Descripción: El prototipo no se termine en el tiempo estimado.}\\
		\hline
		\multicolumn{4}{|p{11cm}|}{\textbf{Refinamiento/contexto:} \par 
			\parbox[p][0.2\textwidth][c]{12.3cm}{
				\begin{itemize}
					\item Subcondición 1: Se presenta retrasos en actividades que se consideraban sencillas.
					\item Subcondición 2: Se lleva más tiempo en las actividades que lo esperado.
				\end{itemize}		
			}		
		} \\
		\hline
		\multicolumn{4}{|p{11cm}|}{\textbf{Reducción/supervisión:} \par
			\parbox[p][0.25\textwidth][c]{12.3cm}{
				\begin{enumerate}
					\item Se clasificará las actividades complejas de las sencillas, para poder atacarlas de acuerdo a su complejidad.
					\item Una vez clasificadas se asignará a cada integrante a realizar las actividades, sí las actividades son de desarrollo, se dividirá por módulos para elaborar cada módulo, reduciendo el grado de error y poder identificarlo.
				\end{enumerate}		
			}	
		}\\
		\hline
		\multicolumn{4}{|p{12.3cm}|}{\textbf{Gestión/Plan de contingencia/acción:} Se organizarán juntas, en las que se evaluara el avance del prototipo y las actividades faltantes, para determinar un plan de acción, para regularizar los tiempos con el cronograma.}\\
		\hline
	\end{tabular}
	\textbf{\caption{\small{\textbf{Tabla de datos de riesgo 8 \cite{cuarentaycuatro}.}}}}
\end{table}

%---------------------------- Tabla de riesgo 9 ----------------------% 
\begin{table}[htb]
	\centering
	\begin{tabular}{|p{2.75cm}|p{2.75cm}|p{2.75cm}|p{2.75cm}|}
		\hline
		\multicolumn{4}{|c|}{Hoja de información de riesgo}\\
		\hline
		\multicolumn{1}{|c|}{ID: \textbf{R9}} &
		\multicolumn{1}{|c|}{Fecha: 25/Octubre/2016} &
		\multicolumn{1}{|c|}{Probabilidad: 20\%} &
		\multicolumn{1}{|c|}{Impacto: Catastrófico} \\
		\hline
		\multicolumn{4}{|p{12.3cm}|}{Descripción: El dispositivo móvil no cuenta con las características especificadas para el proyecto.}\\
		\hline
		\multicolumn{4}{|p{11cm}|}{\textbf{Refinamiento/contexto:} \par 
			\parbox[p][0.2\textwidth][c]{12.3cm}{
				\begin{itemize}
					\item Subcondición 1: La aplicación móvil no está disponible para el dispositivo móvil.  
					\item Subcondición 2: La versión del sistema operativo no es compatible con la aplicación.
				\end{itemize}		
			}		
		} \\
		\hline
		\multicolumn{4}{|p{11cm}|}{\textbf{Reducción/supervisión:} \par
			\parbox[p][0.2\textwidth][c]{12.3cm}{
				\begin{enumerate}
					\item Se realizarán pruebas en distintos sistemas operativos de la misma familia.
					\item Después de realizar las pruebas, se clasificará los sistemas operativos aceptables para el prototipo. 
				\end{enumerate}		
			}	
		}\\
		\hline
		\multicolumn{4}{|p{12.3cm}|}{\textbf{Gestión/Plan de contingencia/acción:} Se indicarán las tecnologías requeridas para el funcionamiento del prototipo.}\\
		\hline
	\end{tabular}
	\textbf{\caption{\small{\textbf{Tabla de datos de riesgo 9 \cite{cuarentaycuatro}.}}}}
\end{table}

%---------------------------- Tabla de riesgo 10 ----------------------% 
\begin{table}[htb]
	\centering
	\begin{tabular}{|p{2.75cm}|p{2.75cm}|p{2.75cm}|p{2.75cm}|}
		\hline
		\multicolumn{4}{|c|}{Hoja de información de riesgo}\\
		\hline
		\multicolumn{1}{|c|}{ID: \textbf{R10}} &
		\multicolumn{1}{|c|}{Fecha: 25/Octubre/2016} &
		\multicolumn{1}{|c|}{Probabilidad: 10\%} &
		\multicolumn{1}{|c|}{Impacto: Critico} \\
		\hline
		\multicolumn{4}{|c|}{Descripción: El usuario directo arruine la pulsera de sensores.}\\
		\hline
		\multicolumn{4}{|p{11cm}|}{\textbf{Refinamiento/contexto:} \par 
			\parbox[p][0.2\textwidth][c]{12cm}{
				\begin{itemize}
					\item Subcondición 1: El usuario indirecto no tiene conocimiento de las actividades que ponen en riesgo el funcionamiento del prototipo. 
					\item Subcondición 2: Se olvida en que momentos utilizar el prototipo.
				\end{itemize}		
			}		
		} \\
		\hline
		\multicolumn{4}{|p{11cm}|}{\textbf{Reducción/supervisión:} \par
			\parbox[p][0.2\textwidth][c]{12cm}{
				\begin{enumerate}
					\item Se estudiará las actividades que realizan las personas de la tercera edad con capacidades limitadas.
					\item Se agrupará las actividades que no afecten el funcionamiento del prototipo y se indicaran en el manual. 
				\end{enumerate}		
			}	
		}\\
		\hline
		\multicolumn{4}{|p{12cm}|}{\textbf{Gestión/Plan de contingencia/acción:} Se pondrán en el manual de usuario, acciones que pongan en riesgo el funcionamiento del prototipo, con el fin de que se tomen en cuenta.}\\
		\hline
	\end{tabular}
	\textbf{\caption{\small{\textbf{Tabla de datos de riesgo 10 \cite{cuarentaycuatro}.}}}}
\end{table}

%---------------------------- Tabla de riesgo 11 ----------------------% 
\begin{table}[htb]
	\centering
	\begin{tabular}{|p{2.75cm}|p{2.75cm}|p{2.75cm}|p{2.75cm}|}
		\hline
		\multicolumn{4}{|c|}{Hoja de información de riesgo}\\
		\hline
		\multicolumn{1}{|c|}{ID: \textbf{R11}} &
		\multicolumn{1}{|c|}{Fecha: 25/Octubre/2016} &
		\multicolumn{1}{|c|}{Probabilidad: 30\%} &
		\multicolumn{1}{|c|}{Impacto: Catastrófico} \\
		\hline
		\multicolumn{4}{|p{12.3cm}|}{Descripción: Se presente un inconveniente en el desarrollo por falta de conocimiento.}\\
		\hline
		\multicolumn{4}{|p{11cm}|}{\textbf{Refinamiento/contexto:} \par 
			\parbox[p][0.23\textwidth][c]{12.3cm}{
				\begin{itemize}
					\item Subcondición 1: No se tiene conocimiento de cómo realizar dichas actividades. 
					\item Subcondición 2: En el desarrollo se presenta una etapa que no se consideró o planifico durante el análisis, por tal motivo se presentan errores en el desarrollo que se desconocen su causa.
				\end{itemize}		
			}		
		} \\
		\hline
		\multicolumn{4}{|p{11cm}|}{\textbf{Reducción/supervisión:} \par
			\parbox[p][0.33\textwidth][c]{12.3cm}{
				\begin{enumerate}
					\item Se realizará una investigación previa para conocer los temas que se requieren para poder desarrollar el prototipo.
					\item Se investigará los cursos, talleres y libros, que contengan este conocimiento.
					\item Se enfocará el equipo en entender dicho conocimiento, mediante prácticas. 
				\end{enumerate}		
			}	
		}\\
		\hline
		\multicolumn{4}{|p{12.3cm}|}{\textbf{Gestión/Plan de contingencia/acción:} Se organizarán juntas con el fin de discutir los temas desconocidos, para después tomar cursos o talleres que nos permitan cubrir ese conocimiento.}\\
		\hline
	\end{tabular}
	\textbf{\caption{\small{\textbf{Tabla de datos de riesgo 11 \cite{cuarentaycuatro}.}}}}
\end{table}

%---------------------------- Tabla de riesgo 12 ----------------------% 
\begin{table}[htb]
	\centering
	\begin{tabular}{|p{2.75cm}|p{2.75cm}|p{2.75cm}|p{2.75cm}|}
		\hline
		\multicolumn{4}{|c|}{Hoja de información de riesgo}\\
		\hline
		\multicolumn{1}{|c|}{ID: \textbf{R12}} &
		\multicolumn{1}{|c|}{Fecha: 25/Octubre/2016} &
		\multicolumn{1}{|c|}{Probabilidad: 30\%} &
		\multicolumn{1}{|c|}{Impacto: Despreciable} \\
		\hline
		\multicolumn{4}{|p{12.3cm}|}{Descripción: La alimentación suministrada por las pilas hacia los sensores no sea suficiente para realizar las pruebas mínimas.}\\
		\hline
		\multicolumn{4}{|p{11cm}|}{\textbf{Refinamiento/contexto:} \par 
			\parbox[p][0.2\textwidth][c]{12.3cm}{
				\begin{itemize}
					\item Subcondición 1: La alimentación no permite llevar acabo las mediciones necesarias para el usuario. 
					\item Subcondición 2: La alimentación no cubre la demanda de corriente de los dispositivos.
				\end{itemize}		
			}		
		} \\
		\hline
		\multicolumn{4}{|p{11cm}|}{\textbf{Reducción/supervisión:} \par
			\parbox[p][0.3\textwidth][c]{12.3cm}{
				\begin{enumerate}
					\item Se observará la corriente total consumida por los dispositivos electrónicos.
					\item Determinar el arreglo de pilas que cubra la demanda.
					\item Asignar una frecuencia de mediciones, para cambiar el estado de los sensores y microcontroladores a un modo ahorro de energía. 
				\end{enumerate}		
			}	
		}\\
		\hline
		\multicolumn{4}{|p{12.3cm}|}{\textbf{Gestión/Plan de contingencia/acción:} Se realizarán pruebas una vez que se tenga implementado todo el prototipo, con el fin de analizar la corriente que esté consume, y determinar el arreglo de pilas que se debe utilizar.}\\
		\hline
	\end{tabular}
	\textbf{\caption{\small{\textbf{Tabla de datos de riesgo 12 \cite{cuarentaycuatro}.}}}}
\end{table}

%---------------------------- Tabla de riesgo 13 ----------------------% 
\begin{table}[htb]
	\centering
	\begin{tabular}{|p{2.75cm}|p{2.75cm}|p{2.75cm}|p{2.75cm}|}
		\hline
		\multicolumn{4}{|c|}{Hoja de información de riesgo}\\
		\hline
		\multicolumn{1}{|c|}{ID: \textbf{R13}} &
		\multicolumn{1}{|c|}{Fecha: 25/Octubre/2016} &
		\multicolumn{1}{|c|}{Probabilidad: 10\%} &
		\multicolumn{1}{|c|}{Impacto: Critico} \\
		\hline
		\multicolumn{4}{|c|}{Descripción: El microcontrolador se arruina.}\\
		\hline
		\multicolumn{4}{|p{11cm}|}{\textbf{Refinamiento/contexto:} \par 
			\parbox[p][0.2\textwidth][c]{12cm}{
				\begin{itemize}
					\item Subcondición 1: La temperatura soportada por el microcontrolador es sobrepasada y se arruina. 
					\item Subcondición 2: Se realizó una conexión externa de manera errónea.
				\end{itemize}		
			}		
		} \\
		\hline
		\multicolumn{4}{|p{11cm}|}{\textbf{Reducción/supervisión:} \par
			\parbox[p][0.3\textwidth][c]{12cm}{
				\begin{enumerate}
					\item Cada integrante se documentará mediante la hoja de especificación.
					\item Se analizará el entorno en el que se realizan las pruebas, para determinar si es el apropiado.
					\item Antes de alimentar los circuitos se observará si las conexiones externas son correctas.
				\end{enumerate}		
			}	
		}\\
		\hline
		\multicolumn{4}{|p{12cm}|}{\textbf{Gestión/Plan de contingencia/acción:} Comprar varios repuestos para poder cambiarlos, además de analizar el manual para determinar buenas prácticas.}\\
		\hline
	\end{tabular}
	\textbf{\caption{\small{\textbf{Tabla de datos de riesgo 13 \cite{cuarentaycuatro}.}}}}
\end{table}

%---------------------------- Tabla de riesgo 14 ----------------------% 
\begin{table}[htb]
	\centering
	\begin{tabular}{|p{2.75cm}|p{2.75cm}|p{2.75cm}|p{2.75cm}|}
		\hline
		\multicolumn{4}{|c|}{Hoja de información de riesgo}\\
		\hline
		\multicolumn{1}{|c|}{ID: \textbf{R14}} &
		\multicolumn{1}{|c|}{Fecha: 25/Octubre/2016} &
		\multicolumn{1}{|c|}{Probabilidad: 20\%} &
		\multicolumn{1}{|c|}{Impacto: Critico} \\
		\hline
		\multicolumn{4}{|c|}{Descripción: El prototipo no cumple las expectativas del cliente.}\\
		\hline
		\multicolumn{4}{|p{11cm}|}{\textbf{Refinamiento/contexto:} \par 
			\parbox[p][0.16\textwidth][c]{12cm}{
				\begin{itemize}
					\item Subcondición 1: El prototipo no es confortable para el cliente.
					\item Subcondición 2: El cliente no está convencido de la solución que ofrece nuestro prototipo.
				\end{itemize}		
			}		
		} \\
		\hline
		\multicolumn{4}{|p{11cm}|}{\textbf{Reducción/supervisión:} \par
			\parbox[p][0.26\textwidth][c]{12cm}{
				\begin{enumerate}
					\item Se tratará de realizar el prototipo lo más pequeño que se pueda.
					\item Se analizará el diseño del prototipo con el fin de que no sea invasivo.
					\item Se estudiarán las actividades del usuario indirecto para estimar el tamaño del accesorio que se debe desarrollar.
				\end{enumerate}		
			}	
		}\\
		\hline
		\multicolumn{4}{|p{12cm}|}{\textbf{Gestión/Plan de contingencia/acción:} Se pensará en las dificultades del cliente indirecto, con el fin de reducir las molestias que pudiese presentar el prototipo a la hora de ser usado.}\\
		\hline
	\end{tabular}
	\textbf{\caption{\small{\textbf{Tabla de datos de riesgo 14 \cite{cuarentaycuatro}.}}}}
\end{table}

%---------------------------- Tabla de riesgo 15 ----------------------% 
\begin{table}[htb]
	\centering
	\begin{tabular}{|p{2.75cm}|p{2.75cm}|p{2.75cm}|p{2.75cm}|}
		\hline
		\multicolumn{4}{|c|}{Hoja de información de riesgo}\\
		\hline
		\multicolumn{1}{|c|}{ID: \textbf{R15}} &
		\multicolumn{1}{|c|}{Fecha: 25/Octubre/2016} &
		\multicolumn{1}{|c|}{Probabilidad: 10\%} &
		\multicolumn{1}{|c|}{Impacto: Critico} \\
		\hline
		\multicolumn{4}{|c|}{Descripción: Los sensores no se encuentre calibrados.}\\
		\hline
		\multicolumn{4}{|p{11cm}|}{\textbf{Refinamiento/contexto:} \par 
			\parbox[p][0.16\textwidth][c]{12cm}{
				\begin{itemize}
					\item Subcondición 1: Los sensores tienen un error en las mediciones.
					\item Subcondición 2: Las mediciones no coinciden con el usuario.
				\end{itemize}		
			}		
		} \\
		\hline
		\multicolumn{4}{|p{11cm}|}{\textbf{Reducción/supervisión:} \par
			\parbox[p][0.16\textwidth][c]{12cm}{
				\begin{enumerate}
					\item Se analiza al cliente para determinar las configuraciones. 
					\item Se asignan los parámetros iniciales para cada sensor.
				\end{enumerate}		
			}	
		}\\
		\hline
		\multicolumn{4}{|p{12cm}|}{\textbf{Gestión/Plan de contingencia/acción:} Se implementará un estudio del usuario final con el fin de estudiar las variables, que pudiese presentar su condición actual, se considerará el factor de emisión de su frecuencia cardíaca acorde a su edad.}\\
		\hline
	\end{tabular}
	\textbf{\caption{\small{\textbf{Tabla de datos de riesgo 15 \cite{cuarentaycuatro}.}}}}
\end{table}

\clearpage
