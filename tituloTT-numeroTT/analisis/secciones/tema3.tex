\section{Análisis de requerimientos}

Los requerimientos funcionales son las soluciones que debe cumplir el prototipo para cubrir los objetivos generales. \\

El origen de los requerimientos del prototipo, derivan por una parte del aumento de la población de adultos mayores, las vulnerabilidades que presentan, brindando herramientas que faciliten su cuidado y atención, sin descuidar las actividades que el jefe del hogar debe realizar para solventar los gastos, está a su vez debe tener un precio accesible. \\

La problemática se identificó de informes nacionales e internacionales dedicados al estudio del núcleo familiar, aumento de la población, el riesgo y capacidades diferentes que presentan las personas de la tercera edad y el uso del tiempo de hombres y mujeres de edad avanzada. Los informes fueron obtenidos de organizaciones como OMS, INEGI y Banco Mundial. \\

\begin{itemize}
	\item \textbf{Requerimientos funcionales:}
\end{itemize}

A continuación, se muestran los requerimientos funcionales: \\

\begin{table}
	\centering
	\begin{tabular}{|p{2cm}|p{3cm}|p{6cm}|}
		\hline
		Identificador & Nombre & Descripción \\
		\hline \hline
		RF1 & Alimentar el prototipo & Alimentar el prototipo con 3v o 5v \\
		\hline
		RF2 & Configuración del prototipo & Configurar las características del usuario directo en el microcontrolador así como la frecuencia de lecturas que se realizaran al usuario directo \\
		\hline
		RF3 & Lectura de datos & Obtener las magnitudes medidas por el sensor de temperatura, ritmo cardíaco y acelerómetro \\		
		\hline
		RF4 & Evaluación de la temperatura & Verificar si la temperatura del usuario directo se encuentra entre $(36\ldotp5^{\circ}C)$ y $(37\ldotp2^{\circ}C)$ (de acuerdo a la OMS) \\
		\hline
		RF5 & Evaluación de ritmo cardíaco & Verificar el ritmo cardíaco del usuario directo se encuentra en un rango aceptable de acuerdo a su edad (basadas en mediciones de instituciones médicas) \\
		\hline
		RF6 & Evaluación de la aceleración & Verificar si la magnitud arrojada por el acelerómetro es normal en una persona de la tercera edad \\
		\hline
		RF7 & Emisión de alertas & Informar al usuario indirecto si los parámetros obtenidos han sobrepasado los rangos aceptables \\
		\hline
		RF8 & Prueba de comunicación & Comprobar la comunicación entre los sensores y el microcontrolador, el dispositivo y la aplicación \\
		\hline
		RF9 & Estado del usuario directo & Muestra la aplicación los últimos valores obtenidos por los sensores \\
		\hline
		RF10 & Historial de variables & Almacenar las lecturas obtenidas por los sensores \\
		\hline
	\end{tabular}
	\textbf{\caption{\small{\textbf{Requerimientos funcionales.}}}}
\end{table}

\newpage

\begin{itemize}
	\item \textbf{Requerimientos no funcionales:}
\end{itemize}

Los requerimientos no funcionales no tienen que ver con los objetivos generales de manera directa, pero no se involucran con las funciones primordiales del prototipo, y se identificaron los siguientes: \\

\begin{table}
	\centering
	\begin{tabular}{|p{2cm}|p{3cm}|p{6cm}|}
		\hline
		Identificador & Nombre & Descripción\\
		\hline \hline
		RF1 & Costo & El costo debe ser accesible para el usuario final \\
		\hline
		RF2 & Interfaz & La interfaz debe ser amigable para el usuario \\
		\hline
		RF3 & Alimentación del prototipo & El prototipo debe tener un consumo mínimo de energía \\		
		\hline
		RF4 & Confort del prototipo	& El dispositivo no debe ser incómodo para el usuario directo \\
		\hline
		RF5 & Estabilidad & El prototipo debe ser constante al tiempo que debe emitir las mediciones, conforme se haya configurado la frecuencia de estás \\
		\hline
	\end{tabular}
	\textbf{\caption{\small{\textbf{Requerimientos no funcionales.}}}}
\end{table}

\begin{itemize}
	\item \textbf{Actores del sistema:} 
\end{itemize}

\begin{enumerate}
	\item \textbf{Usuario Directo:} Es la persona portadora del dispositivo, a la cual se estará monitoreando y de quien se obtendrán las lecturas de: temperatura, ritmo cardíaco y caídas.
	\item \textbf{Usuario Indirecto:} Es la persona que tendrá acceso a la aplicación móvil y a quien se le harán las notificaciones en caso de que el usuario directo necesite algún tipo de ayuda
\end{enumerate}
