\chapter{Introducci\'on}

La Clasificación Internacional del Funcionamiento de la Discapacidad y de la Salud (CIF) define la discapacidad como un término genérico que engloba deficiencias, limitaciones de actividad y restricciones para la participación social. La discapacidad forma parte de la condición humana: “… casi todas las personas sufrirán algún tipo de discapacidad transitoria o permanente en algún momento de su vida, y las que lleguen a la senilidad experimentarán dificultades crecientes de funcionamiento”. \\

De acuerdo a los informes de la Organización Mundial de la Salud (OMS), se estima que más de mil millones de personas viven con algún tipo de discapacidad; es decir, alrededor del 15\% de la población mundial (según las estimaciones de la población mundial en 2010) \cite{uno}. En relación con lo anterior, en el país existen 31.5 millones de hogares, de ellos 6.1 millones reportan que existe al menos una persona con discapacidad; es decir, en 19 de cada 100 hogares vive una persona que presenta alguna dificultad. \\

En el 2013 6.6\% de la población mexicana reportó tener una discapacidad, siendo en su mayoría las personas adultos mayores, con 51.4\%, informó el Instituto Nacional de Estadística y Geografía (INEGI). Por otra parte, la Encuesta Nacional de Ingresos y Gastos de los Hogares 2012 (ENIGH 2012), dicho porcentaje de la población del país presentó dificultad (discapacidad) para realizar al menos una de las actividades como: caminar, ver, escuchar, hablar o comunicarse, poner atención o aprender, atender el cuidado personal y mental \cite{dos}. Mientras que en los adultos mayores la enfermedad y la edad es el factor detonante. En los adultos mayores, el 50.9\% de las discapacidades se tienen por origen de la edad avanzada. \\
De acuerdo con las proyecciones del Consejo Nacional de Población (CONAPO), hasta 2010 la población en el Distrito Federal de 65 años en adelante representa el 7.9\% del total. Para 2030, serán los “viejitos” del futuro y demandarán productos y servicios especiales para ellos. \\

Aunado a este factor, "las Instituciones del Gobierno y la oferta de salud entre hospitales y personal especializado no son suficientes para atender a la población que sufre de alguna discapacidad, por otra parte del costo por estos servicios es elevado oscilando desde \$7,000 M.N. a \$30,000 M.N. mensuales \cite{tres}. \\

En la actualidad existen programas sociales por parte del Gobierno de la Ciudad de México, uno de ellos es “Médico en tu casa” este programa fue creado para reducir índices de mortalidad por embarazos en Iztapalapa y Gustavo A. Madero. El cual consiste en enviar médicos para que atiendan a mujeres embarazadas, adultos mayores y niños. El principal objetivo de este programa es brindar atención a la población vulnerable, principalmente adultos mayores, discapacitados, enfermos terminales, así como disminuir el índice de mortalidad materna-infantil en la capital. \\

Sumado a lo anterior es necesario utilizar herramientas que nos permitan facilitar el cuidado de las personas con discapacidad, dando una opción más accesible para la población que no cuente con recursos necesarios para pagar algún servicio de cuidado y atención. \\

La idea de utilizar este tipo de herramientas es facilitar el cuidado y la atención de las personas con alguna discapacidad, teniendo en cuenta que aún pueden realizar actividades de la vida cotidiana, además de estar pendientes de los momentos exactos en los que la persona presente una situación delicada en su estado de salud. Una de estas alternativas es utilizar sensores que permitan supervisar las vulnerabilidades que presente, buscando aprovechar al máximo el tiempo de los familiares sin descuidar la atención que requiere el familiar con discapacidad, sin dejar de lado la necesidad de reducir los precios que conlleva el cuidado de las personas con discapacidad, ya que con esta propuesta permitirá a los familiares que no cuenten con los recursos para este tipo de servicios, puedan usarla como apoyo al cuidado de sus familiares, siendo el costo del prototipo más accesible, pensando en que se realice con un menor monto al requerido por las instituciones que brindan estos servicios. \\



