\section{Justificación}

Como se mencionó anteriormente la discapacidad en el país ha ido en aumento en los últimos años debido a la transición demográfica, la reducción de mortalidad-natalidad y accidentes ocasionados en el trabajo. La mayor parte de la población discapacitada son los adultos mayores de 60 años. Es decir, la enfermedad o la edad avanzada son las principales causas para todos los tipos de discapacidad considerados. Con el desarrollo de este prototipo se pretende dar atención a una necesidad social que cada vez va en aumento, debido a los altos costos que implica contratar un servicio especializado que aún no se encuentra plenamente desarrollado. Aunque es necesario puntualizar que este prototipo solo supervisará aquellos casos en los que el adulto mayor no se encuentre postrado en una cama. \\


\begin{table}[htb]
	\centering
	\begin{tabular}{|p{4cm}|p{1.5cm}|p{1.5cm}|p{1.5cm}|p{1.5cm}|}
		\hline
		Tipos de discapacidad & \multicolumn{4}{c|}{Grupos de edad} \\
		\cline{2-5} 
		& 0 a 14 años & 15 a 29 años & 30 a 59 años & 60 años o más \\
		\hline \hline
		Caminar, subir o bajar usando sus piernas. & 36.2 & 32.1 & 56.2 & 81.3 \\ \cline{2-5}
		\hline 
		Ver (aunque usen lentes). & 26.9 & 44.6	& 58.2 & 67.2 \\ \cline{2-5}
		\hline 
		Mover o usar sus brazos o manos. & 14.1 & 18.2 & 28.5 &	42.7 \\ \cline{2-5}
		\hline 
		Aprender, recordar o concentrarse. & 40.8 & 31.5 & 32.1 & 44.6 \\ \cline{2-5}
		\hline 
		Escuchar (aunque usen aparato auditivo). & 13.4	& 18.5 & 24.2 & 46.9 \\ \cline{2-5}
		\hline 
		Bañarse, vestirse o comer. & 37.4 & 16.4 & 14.5 & 29.3 \\ \cline{2-5}
		\hline 
		Hablar o comunicarse. & 45.6 & 28.5 & 13.4 & 14.0 \\ \cline{2-5}
		\hline 
		Problemas emocionales o mentales. & 26.6 & 28.0 & 20.1 & 16.3  \\ \cline{2-5}
		\hline
	\end{tabular}
	\textbf{\caption{\small{\textbf{Porcentaje de población con discapacidad, por tipo de discapacidad según grupos de edad en 2014 \cite{diez}.}}}}
	\label{Justificacion:Tabla1.1}
\end{table}

Por lo que se requiere de servicios especializados en salud enfocados a este sector, en la actualidad existen programas para atender este tipo de problemáticas por mencionar algunos provenientes del Gobierno de la Ciudad de México como: 

\begin{itemize}
	\item Médico en tu casa.
	\item Programa Nacional para el Desarrollo y la Inclusión de las personas con discapacidad.
\end{itemize}

Por parte de la iniciativa Privada existen servicios para este tipo de necesidad como:

\begin{itemize}
	\item Cuidado personalizado (Homewatch Care Givers).
	\item Estancia (La Casa de Las Lunas).
\end{itemize}

Teniendo costos elevados que desequilibran la economía familiar, tal y como se mencionó en la introducción de la investigación. Sin embargo, los programas gubernamentales no logran cubrir la demanda para estos servicios y por parte de la iniciativa privada sus costos son elevados, como consecuencia no toda la población cuenta con los recursos suficientes para pagar un servicio como esté. \\
 
Por lo que se propone un prototipo para ayudar a personas con determinada discapacidad, implementando una serie de alertas que permitan a los familiares estar al tanto de la situación del discapacitado, por otro lado, se busca que este prototipo sea portable y económico, de esta manera reducir los costos que significa este tipo de cuidados, siendo una alternativa o complemento de los servicios ya existentes. \\

Para cubrir la necesidad de la población con bajos recursos, se piensa diseñar una tarjeta de propósito específico a partir de sensores y microcontroladores adecuados con el fin de reducir los costos, ya que estos son muy accesibles en el mercado, utilizados para múltiples aplicaciones permitiendo que el prototipo sea escalable y portable. \\

Entre las diferencias con respecto a lo ya existente, son las alertas instantáneas emitidas por los sensores logrando mantener al familiar informado de la situación en la que se encuentra la persona con discapacidad. Cabe mencionar que los sistemas de monitoreo existentes están diseñados para que una persona esté siempre monitoreando la situación del discapacitado.


 